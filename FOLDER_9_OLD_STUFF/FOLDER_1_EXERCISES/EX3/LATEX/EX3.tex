\documentclass{article}
\usepackage{hyperref}
\begin{document}
\title{Exercise 3}
\date{Due 12/5/2017}
\maketitle

\section{Introduction}
This exercise checks whether a simple arithmetic expression is valid or not.
We will use the grammar in Table \ref{Table_Grammar},
and apply the SLR$(1)$ parsing scheme on top of FLEX.
\begin{table}[h]
\centering
\begin{tabular}{ l l l l l l l}
  $S \rightarrow E\$$ & & $E \rightarrow E+T$ & & $T \rightarrow T*F$ & & $F \rightarrow int$ \\
                      & & $E \rightarrow E-T$ & & $T \rightarrow T/F$ & & $F \rightarrow (E)$ \\
                      & & $E \rightarrow   T$ & & $T \rightarrow   F$ & &                     \\
\end{tabular}
\caption{The grammar for a simple calculator with operator precedence. \label{Table_Grammar}}
\end{table}
You need not perform the actual computation, but rather say
whether it is a valid arithmetic computation.
For instance, $4+5*6$ is valid,
but $4+5*))99$ is \textit{not} valid.

\section{Algorithm Outline}
The shift reduce SLR$(1)$ algorithm was discussed thoroughly in class.
It can be roughly described as having two modules:
the state transition diagram which you have to write as a two dimensional table,
and the stack which accommodates the relevant history of encountered tokens, reduced variables and states. At each step the state at the top of the stack, and the next token
ahead imply an action which can be one of the following: shift, reduce, accept or error.
Suppose, for the sake of explanation, that state $7$ is currently at the top of the stack, and the next token ahead is $+$. \\ \\
If \verb"TABLE"$[7][+]$ = ``s3" then it means the algorithm should
perform a shift action, push the element $(+,3)$ on top of the stack, and read the next token from the input.\\ \\
If \verb"TABLE"$[7][+]$ = ``r6" then it means the algorithm should
perform a reduce action for rule $6$, pop from the stack as many elements as
the length of the right hand side of rule $6$, and push the left hand side variable of rule $6$ (together with the next state) to the top of the stack.\\ \\
If \verb"TABLE"$[7][+]$ = ``" (the empty string) then it means the algorithm should
print some kind of error message and terminate.\\ \\
If \verb"TABLE"$[7][+]$ = ``a" (accept) then it means the input is correct.\\ \\

\section{Input}
The input for this exercise is a single text file that contains the arithmetic expression.

\section{Output}
The output is a single text file that should contain a single word:
either OK when the expression is valid, or FAIL otherwise. 

\section{Submission Guidelines}
The code for this exercise resides as usual in subdirectory EX3.
The file Table.c should be the only file you change.
Please submit your exercise in your GitHub repository under COMPILATION/EX3,
and have a makefile there to build a runnable program called calc.
To avoid the pollution of EX3, please remove all *.o files once the target is built.
The next paragraph describes the execution of calc.

\paragraph{Execution parameters}
calc recevies $2$ input file names:
\begin{table}[h]
\centering
\begin{tabular}{ l }
  Input.txt  \\
  Output.txt \\
\end{tabular}
\end{table}


\end{document}
